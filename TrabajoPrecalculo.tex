\documentclass[10pt,a4paper]{article}
\usepackage[utf8]{inputenc}
\usepackage[spanish]{babel}
\usepackage{xcolor}
\usepackage[left=2cm,right=2cm,top=2cm,bottom=2cm]{geometry}
\usepackage{multicol}
\usepackage{xfrac}
\usepackage{enumitem}
\author{Darien R Pajaro}
\usepackage{fancyhdr}
\usepackage{titlesec}

\pagestyle{fancy}
\rhead[]{2.5 \ \ Propiedades Formales de los Números Reales}%Cabeza de Pagina%
\begin{document}
%Inicio De La primera pagina%
\begin{flushleft}
\textcolor{red}{\textit{def. campo.}} en Matemáticas avanzadas se toman como axiomas de un sistema abstracto llamado campo. Por lo tanto, podemos decir que los números reales forman un campo.
\end{flushleft}
\section*{\textcolor{red}{Problemas de la sección 2.5}}

\begin{flushleft}
{\bfseries Adición}
\end{flushleft}.
En los problemas del 1 al 4, utiliza las leyes conmutativas y asociativas para establecer la veracidad de la afirmación dada. Realiza tu prueba siguiendo el ejemplo proporcionado para el teorema 2. 
%Ejercicios del 1 al 6%
\begin{multicols}{2}
\begin{enumerate}[label=\textbf{\arabic*.}]
  \item ${4+2+7=2+7+4}$
  \item ${1+5+9=9+5+1}$
  \item ${a+b+c=c+a+b}$
  \item ${a+b+c=b+c+a}$
  \item defina ${a+b+c+d+e}$. 
  \item asumiendo que ${a+b+c+d+e}$ fue definida (Problema 5), defina ${a+b+c+d+e+f}$.
  \item Halle el inverso aditivo de cada uno de los siguientes números: ${ 4,-2,\pi,0,-\sqrt{5}}$.
  \item Halle el inverso aditivo de cada uno de los siguientes números: ${ -6,2,\sfrac{3}{4},-0,3}$.
  \item Halle el valor absoluto de cada uno de los siguientes números: ${ 7,-3,0,\sfrac{2}{3},-\sfrac{1}{2}}$
   \item Halle el valor absoluto de cada uno de los siguientes números: ${ -8,-0,\pi^{2},\sfrac{3}{5},-5}$\\
\end{enumerate}
\end{multicols}
en los problemas del 1 al 6 evalúe la expresión dada
%Ejercicios del 11 al 16%
\begin{enumerate}[label=\textbf{\arabic*.}]
 \setcounter{enumi}{10}
    \item ${[4+(2-5)]-[6-(5-2)]}$
    \item ${[3-(-2+4)]-[12+(8-4)]}$
    \item ${[7+(1-8)]+[3-(6-2)]}$
    \item ${[(-12+4)+7]+[-7+(4-8)]}$
    \item ${\lbrace[(-2+4)-(12+6)]-[-15+8]\rbrace-24}$
    \item ${\lbrace[(5-9)-(6-12)]-[8+5]\rbrace+14}$    
\end{enumerate}
{\bfseries Multiplicación}.
%Ejercicios del 17 al 18%
\begin{enumerate}[label=\textbf{\arabic*.}]
 \setcounter{enumi}{16}
    \item Defina ${a\times b\times c}$.
    \item Asumiendo que ${a\times b\times c}$ está definida (Problema 17), defina ${a\times b\times c \times d}$
\end{enumerate}.
%Ejercicios del 19 al 24  %
En los problemas del 19 al 22, utiliza las leyes conmutativas y asociativas para establecer la veracidad de la afirmación dada.
\begin{multicols}{2}
\begin{enumerate}[label=\textbf{\arabic*.}]
\setcounter{enumi}{18}
	\item ${4\times 5\times 7 = 7 \times 5 \times 4}$
	\item ${6\times 2\times 3 = 6 \times 3 \times 2}$
	\item ${a\times b\times c = b \times a \times c }$
	\item ${a\times b\times c = c \times a \times b}$
\end{enumerate}
\end{multicols}
\begin{enumerate}[label=\textbf{\arabic*.}]
\setcounter{enumi}{22}
\item encuentre el inverso multiplicativo de cada uno de los siguientes numeros: ${\sfrac{1}{3},-4.-\sfrac{3}{4},1,0}$.
\item encuentre el inverso multiplicativo de cada uno de los siguientes numeros: ${2,-\pi.\sfrac{1}{2},\sqrt{5},\sfrac{7}{3}}$.
\end{enumerate}
En los problemas 25 a 30 evalúe la expresión dada
\begin{enumerate}[label=\textbf{\arabic*.}]
\setcounter{enumi}{24}
	\item ${(-2)[3(4-2)+6]+5[-2(-3+8)+9]}$
	\item ${5[-9(5-2)+4]-4[3(4+8)-45]}$
	\item ${3[8(-2+4)-5(7-9)]+5[(4-8)6-(9-4)3]}$
	\item ${-6[3(7+6)-4(3-8)]-4[7-3]9-16}$
	\item ${4\left\lbrace[-6(3-7)+5(6+3)]-16(2-3) \right\rbrace-11}$
	\item ${-3\left\lbrace[7(9-4)-6(3+7)]+5(-4+8) \right\rbrace+9}$
		
\end{enumerate}
%Fin de la primera pagina%
%Inicio de la segunda pagina%
\newpage

\begin{flushleft}
{\bfseries Resta y División}.
\end{flushleft}
\begin{enumerate}[label=\textbf{\arabic*.}]
\setcounter{enumi}{30}
\item ¿Se cumple la ley conmutativa para la resta de números reales?
\item ¿Se cumple la ley conmutativa para la división de números reales?
\item ¿Se cumple la ley asociativa para la resta de números reales?
\item ¿Se cumple la ley asociativa para la división de números reales?
\item ¿Existe un elemento neutro para la resta? En caso afirmativo, ¿cuál es?
\item ¿Existe un elemento neutro para la división? En caso afirmativo, ¿cuál es?
\end{enumerate}
{\bfseries Cero}.
\begin{enumerate}[label=\textbf{\arabic*.}]%Ejercicios del 37 al 42%
\setcounter{enumi}{36}
\item ¿Qué interpretación se debe dar a cada uno de los siguientes enunciados?\\

${\displaystyle \Large \ \hspace{1.5cm}\ \frac{4}{0} \ \hspace{1cm}\ \frac{0}{4} \ \hspace{1cm}\ \frac{4}{4} \ \hspace{1cm}\ \frac{0}{\sfrac{1}{4}} \ \hspace{1cm}\ \frac{0}{0}}$

\item ¿Qué interpretación se debe dar a cada uno de los siguientes enunciados?\\

${\displaystyle \Large \ \hspace{1.5cm}\ \frac{2}{0} \ \hspace{1cm}\ \frac{0}{2} \ \hspace{1cm}\ \frac{2}{2} \ \hspace{1cm}\ \frac{0}{\sfrac{1}{2}} \ \hspace{1cm}\ \frac{0}{0}}$

\item ¿Para que valores de x las siguientes expresiones carecen de sentido?\\

${\displaystyle \Large \ \hspace{1.5cm}\ \frac{2x-1}{x+1} \ \hspace{1cm}\ \frac{5}{x} \ \hspace{1cm}\ \frac{x+4}{x-2} \ \hspace{1cm}\ \frac{0}{x^{2}+5} \ \hspace{1cm}\ \frac{4}{x^{2}-5x+4}}$

\item ¿Para que valores de x las siguientes expresiones carecen de sentido?\\

${\displaystyle \Large \ \hspace{1.5cm}\ \frac{x}{3} \ \hspace{1cm}\ \frac{3x+2}{x-1} \ \hspace{1cm}\ \frac{x-3}{x^{2}+8} \ \hspace{1cm}\ \frac{3}{x} \ \hspace{1cm}\ \frac{2}{x^{2}+4x+4}}$





 
\end{enumerate}








\end{document}
